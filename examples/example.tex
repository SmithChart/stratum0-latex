\usepackage[utf8]{inputenc}
\usepackage[ngerman]{babel}
\usepackage{amsmath}
\usepackage{tikz}
\usepackage{etex}

\title[kurzer Titel]{Hier steht der Titel dieser Präsentation\\ (max 2 Zeilen)}
\subtitle{Beschreibender Untertitel, falls gewünscht (1 Zeile)}
\author[Kurzname]{\textless{}Name\textgreater}
\date{\today}

\begin{document}

\begin{frame}[plain]
  \titlepage
\end{frame}


\section{Einleitung}

\begin{frame}{Titel}
  \begin{block}{Block}
    \begin{itemize}
      \item mit
      \item etwas
      \begin{itemize}
        \item Inhalt
        \item Aufzählungen
        \item mehr\dots
      \end{itemize}
    \end{itemize}
  \end{block}
\end{frame}


\begin{frame}
\begin{enumerate}
\item irgendwas
\item zählt
  \begin{enumerate}
  \item immer
  \item wieder
  \end{enumerate}
\end{enumerate}
\end{frame}

\pictureframe{../rect.png}

\picturetitleframe{Rechteck}{../rect.png}

\section{Ende}

\begin{frame}{title}
  \begin{tikzpicture}%
    \draw[ultra thin,red] (0,0) rectangle (\textwidth,\textheight);
    \node[below right] at (1em,\textheight-1em) {content area};
  \end{tikzpicture}
\end{frame}

\begin{logoframe}{Ein Titel}
Und Text
\end{logoframe}


\begin{finalframe}{\textless{}Grußformel, 1 Zeile\textgreater}
\textless{}Name\textgreater\\
\textless{}Kontaktdaten\textgreater
\end{finalframe}

\end{document}
